\documentclass[12pt]{article}
\usepackage[russian]{babel}
\usepackage{amsmath}
\usepackage{amssymb}
\usepackage[papersize={145.5mm,254mm}, right=0.5cm, left=0.7cm, top=2.2cm, bottom=1.2cm]{geometry}
\usepackage{titlesec}
\usepackage{fancyhdr}
\usepackage{pgfplots}
\usepackage{tikz}
\pgfplotsset{width=13cm,compat=1.18}

\begin{document}

\setlength{\headsep}{0.4cm}
\setlength{\parindent}{0.4cm}

\markright{Гл. 1 Введение}

\pagestyle{fancy}
\fancyhf{} 
\fancyhead[L]{\fontsize{10pt}{10pt}\selectfont 26}
\fancyhead[C]{\fontsize{10pt}{10pt}\selectfont\itshape\rightmark}

\fontsize{12pt}{12pt}
\selectfont

{\noindentПолагая в этом тождестве $x = 1,2,\dots ,n$ и складывая почленно получаемые равенства, находим
$$\sum_{k=1}^{n} ((k+1)^3-k^3) = 3 \sum_{k=1}^{n} k^2 + 3 \sum_{k=1}^{n} k + n.$$
Так как $$\sum_{k=1}^{n} k = \frac{n(n+1)}{2} ,$$
то, используя формулу (1), получаем $$(n+1)^3 - 1=3S_n + \frac{3}{2} n(n+1) + n,$$
откуда $$S_n = \frac{1}{6} (2n^3 + 3n^2 + n) = \frac{1}{6} n(n+1)(2n+1).$$
Итак,}

\begin{equation}
1^2 + 2^2 + \ldots + n^2 = \frac{n(n+1)(2n+1)}{6}. \blacktriangle \tag{7}
\end{equation}

{З а м е ч а н и е.} {В § 2 (пример 4) равенство (7) было доказано методом индукции.\\}

{П р и м е р 3.} {Вычислим сумму $S_n(x)=\sum\limits_{k=1}^{n}\sin kx.$\\
\indent $\blacktriangle$ Рассмотрим равенство $$S_n(x)\cdot 2\sin \frac{x}{2} = \sum_{k=1}^{n} 2 \sin kx \sin \frac{x}{2}$$
Так как $$ 2 \sin kx \sin \frac{x}{2} = \cos (k - \frac{1}{2})x - \cos (k+\frac{1}{2})x, $$\\
то по формуле (1) находим
$$ S_n(x)\cdot 2 \sin \frac{x}{2} = \cos \frac{x}{2} - \cos (n + \frac{1}{2})x = 2\sin \frac{n+1}{2} x \sin \frac{n}{2}x,$$
откуда
\begin{center}
$S_n(x) = \frac{\sin\frac{n+1}{2}x\sin\frac{n}{2}x}{\sin\frac{x}{2}},$ если $\sin\frac{x}{2}\neq 0;$
\end{center}
если $\sin (x/2) = 0$, то $S_n(x)=0.\blacktriangle$\\
}

{П р и м е р 4.} {Последовательность $\{x_n\}$ задана формулой $x_n=ax_{n-1}+b.$ Выразить через $x_1, a, b$ и $n$:\\} 

{1) $x_n;$\indent 2) $S_n = \sum\limits_{k=1}^{n} x_k.$\\

\indent $\blacktriangle$ 1) Так как $x_k = ax_{k-1}+b, x_{k-1}=ax_{k-2}+b$, то\\
$x_k-x_{k-1}=a(x_{k-1} - x_{k-2})=a^2(x_{k-2}-x_{k-3})=\ldots=a^{k-2}(x_2-x_1),$
}

\newpage
\pagestyle{fancy}
\fancyhf{} 
\fancyhead[R]{\fontsize{10pt}{10pt}\selectfont27}
\fancyhead[C]{\fontsize{10pt}{10pt}\selectfont\itshape§4. Прогрессии. Суммирование. Бином Ньютона}

{\noindentт. е.
$$x_k-x_{k-1}=a^{k-2}(x_2-x_1).$$}

{\noindent Полагая в этой формуле $k=2,3,\ldots,n$ и складывая получаемые равенства, находим
$$\sum_{k=2}^{n}(x_k-x_{k-1})=(x_2-x_1)\sum_{k=2}^{n}a^{k-2},$$
или
$$x_n-x_1=(x_2-x_1)\frac{a^{n-1}-1}{a-1}=((a-1)x_1+b)\frac{a^{n-1}-1}{a-1},$$
откуда
$$x_n=a^{n-1}x_1+b\frac{a^{n-1}-1}{a-1},\indent a\neq 1.$$}

{\noindent При $a=1$ последовательность $\{x\}$ является арифметической прогрессией с разностью $b,$ и поэтому
$$x_n=x_1+(n-1)b.$$}

{2) $S_n=x_1+\sum\limits_{k=2}^{n}x_k=x_1+a\sum\limits_{k=2}^{n}x_{k-1}+(n-1)b,$\\
$$S_n=x_1+a(S_n-x_n)+(n-1)b,$$}

{$S_n(1-a)=x_1-ax_n+(n-1)b=x_1-a^n x_1-ab\frac{a^{n-1}-1}{a-1}+(n-1)b,$\\ откуда $$S_n=\frac{(n-1)b}{1-a} + \frac{ab}{(a-1)^2}(a^{n-1}-1)+\frac{a^n-1}{a-1}x_1, \indent a \neq 1. \blacktriangle$$}

{П р и м е р 5.} {Последовательность $\{x_n\}$ задана формулой $$x_n=(\alpha +\beta)x_{n-1}-\alpha\beta x_{n-2},$$
где $\alpha\beta\neq 0.$ Выразить $\{x_n\}$ через $x_0, x_1, \alpha, \beta$ и $n.$\\
\indent $\blacktriangle$ Исходное равенство можно записать так:
$$x_n=\alpha x_{n-1} = \beta (x_{n-1}-\alpha x_{n-2}).$$
Обозначим $y_n = x_n-\alpha x_{n-1},$ тогда $y_n=\beta y_{n-1},$ откуда $y_n=\beta ^{n-1}y_1,$\\ т.е. $x_n-\alpha x_{n-1} = \beta ^{n-1}y_1,$ или $$x_n=\alpha x_{n-1}+\beta ^{n-1}y_1.$$
Полагая $x_n=\beta ^n z_n,$ получаем
$$z_n=\frac{\alpha}{\beta}z_{n-1}+\frac{y_1}{\beta}.$$
Считая $\alpha \neq \beta$ и используя результат предыдущего примера, находим
$$z_n=(\fraq{\alpha}{\beta})^{n-1}z_1+\frac{(\frac{\alpha}{\beta})^{n-1}-1}{\frac{\alpha}{\beta}-1}\cdot \frac{y_1}{\beta},$$
где $z_1=x_1 / \beta, y_1=x_1-\alpha x_0.$ Отсюда получаем
$$x_n=x_1\frac{\alpha ^n - \beta ^n}{\alpha -\beta}-\alpha \beta x_0\frac{\alpha ^{n-1}-\beta ^{n-1}}{\alpha -\beta},\indent \alpha \neq \beta .$$}

\newpage
\pagestyle{fancy}
\fancyhf{} 
\fancyhead[L]{\fontsize{12pt}{16pt}\selectfont\itshape (additional task)}
\fancyhead[C]{\fontsize{14pt}{16pt}\selectfont\itshape GNUPlot. 3D surface}

{
Изобразим поверхность, заданную формулой:
\begin{center}
    \boxed{\sqrt{x^2 + y^2 + 7}}\\
\end{center}

\vspace{1cm}

\begin{tikzpicture}
    \begin{axis}
    \addplot3[surf]
    {sqrt(x^2 + y^2 + 7)};
    \end{axis}
\end{tikzpicture}

\vspace{1cm}

}

\end{document}
