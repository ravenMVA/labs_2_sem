\documentclass[a4paper]{article}
\usepackage{fullpage}
\usepackage{gnuplottex}
\usepackage[russian]{babel}
\usepackage[inline]{enumitem}
\usepackage{amsmath,amssymb,soul,textcomp}
\makeatletter
    \AddEnumerateCounter{\asbuk}{\@asbuk}
\makeatother
\renewcommand{\labelenumii}{\arabic{enumii})}
\renewcommand{\theenumiii}{\asbuk{enumiii}}
\renewcommand{\labelenumiii}{\theenumiii)}

\begin{document}
\begin{enumerate} [label=\textbf{\arabic*.}]
\setcounter{enumi}{140}
\item \dots держится в $\left[a;b\right]$. Для любых $x, y \in \left[a;b\right], x\not=y$, верно неравнество
\begin{center}
	$\left|f(x)-f(y)\right|<\left|x-y\right|.$
\end{center}
	\begin{enumerate}
		\item
			Доказать, что уравнение $f(x)=x$ имеет и притом единственное решение $c$.
		\item
			Пусть $x_0 \in [a;b], x_n=f(x_{n-1}), n\in N$. Доказать, что
			\begin{enumerate}
				\item
					последовательность $\left\{\left|x_n-c\right|\right\}$ убывает и имеет предел \\ $\lim\limits_{n \to \infty} {\left|x_n-c\right|}=\Delta$;
				\item
					существует последовательность $\left\{x_{n_k}\right\}$, сходящаяся к $d$, равному либо $c+\Delta$, либо $c-\Delta$;
				\item
					$\left|f(d)-c\right|=\Delta$ и $\Delta=0$, т.е. $\lim\limits_{n \to \infty} {x_n}=c$.
			\end{enumerate}
	\end{enumerate}
\item 	\begin{enumerate}
		\item
			Доказать, что уравнение $\tan x=a/x, a>0$, имеет на каждом интервале $(-\pi/2+\pi n; \pi/2+\pi n), n\in N$, одно решение.
		\item
			Пусть $x_n$ - решение уравнения $\tan x=a/x, a>0$, из интервала $(-\pi/2+\pi n; \pi/2+\pi n), n\in N$. Доказать, что
			\begin{center}
				$0<x_n-\pi n<\cfrac{2a}{\pi n+\sqrt{\pi^2 n^2+4a}},n\in N$.
			\end{center}
	\end{enumerate}
\item 	\begin{enumerate}
		\item \makeatletter\def\@currentlabel{1}\makeatother \label{1}
			Доказать, что уравнение $\tan x=ax, a>0$, имеет на каждом интервале $(-\pi/2+\pi n; \pi/2+\pi n), n\in N$, одно решение $x_n$.
		\item \makeatletter\def\@currentlabel{2}\makeatother \label{2}
			Пусть $x_n$ - решение уравнения $\tan x=ax, a>0$, из интервала $(-\pi/2+\pi n; \pi/2+\pi n), n\in N$. Доказать, что
			\begin{center}
				$0<\pi/2+\pi n-x_n<\cfrac{2/a}{\pi/2+\pi n+\sqrt{(\pi/2+\pi n)^2}-4a}$.
			\end{center}
		\item
			Найти $\lim\limits_{n \to \infty} {x_{n+1}-x_n}$, где последовательность $\left\{x_n\right\}$ определена в \ref{1}) и \ref{2}).
	\end{enumerate}
\item 	
	Для последовательности, заданной рекуррентным способом, доказать существование предела и найти его:
\\	\begin{enumerate*} [itemjoin={; \\}]
		\item 
			$x_1\in R, x_{n+1}=\sin x_n, n\in N$
		\item 
			$x_1=0, x_{n+1}=x_n-\sin x_n+1/2, n\in N$
		\item 
			$x_1=\pi/2, x_{n+1}=x_n+\cos x_n-1/2, n\in N$
		\item 
			$x_1\in R, x_{n+1}=\arctg x_n, n\in N$
		\item 
			$x_1=0, x_{n+1}=x_n-\arctg x_n+\pi/4, n\in N$
		\item 
			$x_1=2, x_{n+1}=1+\ln x_n, n\in N$
	\end{enumerate*}
\item 	
	\begin{enumerate}
		\item 
			Пусть $\lim\limits_{n \to \infty} {x_n}=\infty$. Доказать, что $\lim\limits_{n \to \infty} {\left(1+\cfrac1x_n\right)^x_n}=e$.
		\item 
			Пусть $\lim\limits_{n \to \infty} {x_n}=0, x_n \not=0$. Доказать, что $\lim\limits_{n \to \infty} {\left(1+x_n\right)^{1/x_n}}=e$.
	\end{enumerate}
\item
	Пусть $a_1,a_2,b_1,b_2 \in R$. Исследовать на сходимость последовательность
	\begin{center}
		$\left\{\left(\cfrac{a_1n+b_1}{a_2n+b_2}\right)^n\right\}$
	\end{center}
	и найти ее предел, если он существует.
 \item
	Используя непрерывность соответствующих функций, вычислить предел последовательности:
\\	\begin{enumerate*}
		\item 
			$\left\{ \left(1+\cfrac {x_n} n\right)^n\right\}$, если $\lim\limits_{n \to \infty} {x_n}=x\in R$;
\\		\item 
			$\left\{\left(\cos{\cfrac xn}+\lambda\sin{\cfrac x n}\right)^n\right\}$, где $x,\lambda\in R$;
\\		\item 
			$\left\{\left(\cfrac {\sqrt[n]a+\sqrt[n]b}2\right)^n\right\}$, где $a>0, b>0$;
\\		\item 
			$\left\{\left(\left(1+\cfrac an \right)\left(1+\cfrac{2a}n \right)\dots\left(1+\cfrac{ka}n \right)\right)^n\right\}$, где $k \in N, a\in R$;
\\		\item 
			$\left\{\left(\left(1+\cfrac 1{n^2} \right)\left(1+\cfrac 2{n^2} \right)\dots\left(1+\cfrac n{n^2} \right)\right)^n\right\}$, где $k \in N, a\in R$;
\\		\item 
			$\left\{\cos{\cfrac x2}\cos{\cfrac x{2^2}}\dots\cos{\cfrac x{2^n}}\right\}$;
\\		\item
			$\left\{\sin^2{(\pi\sqrt{n^2+n})}\right\}$;
		\item
			$\left\{n-\cfrac1{\sin{(1/n)}}\right\}$;
		\item
			$\left\{n-\ctg\cfrac 1n\right\}$;
\\		\item
			$\left\{\left(cos(x/\sqrt n)\right)^n\right\}$.			
	\end{enumerate*}
\item
	Последовательности $\left\{a_n\right\}$ и $\left\{b_n\right\}$ таковы, что $0<a_n<1, \lim\limits_{n \to \infty} {a_n}=1, 0<b_n<\cfrac \pi2, \cos{b_n}=a_n$. Найти $\lim\limits_{n \to \infty} {\cfrac{b_n}{\sqrt{1-a_n}}}$

ОТВЕТЫ
\setcounter{enumi}{10}
\item \begin{enumerate*}
	\item $f(-1)=-2$;
	\item $f(1)=3/2$;
	\item $f(0)=1/2$;
	\item $f(0)=1$;
	\item $f(0)=1$;
	\item $f(0)=1/2$.	
\end{enumerate*}
\setcounter{enumi}{14}
\item $f(x_0)=0$.
\setcounter{enumi}{17}
\item	 \begin{enumerate}
	\item $x=0,\Delta f(0)=2, x=2, \Delta f(2)=-10$;
	\item $x=-2,\Delta f(-2)=2$;
	\item $x=-2, x=2$ --- точки разрыва 2-го рода;
	\item $x=0$ --- точка разрыва 2-го рода, $x=1, \Delta f(1)=-2$;
	\item $x_n=n, \Delta f(n)=-1, n\in Z$;
	\item $x_n=n, n\in Z$--- точки разрыва 2-го рода;
	\item $x=0, x=1$ --- точка разрыва 2-го рода;	
	\item $x=-1, \Delta f(-1)=0, x=1, \Delta f(1)=-2, x=0$ --- точка разрыва 2-го рода;
	\item $x_n=\pi/2+\pi n, n \in Z$, --- точки разрыва 2-го рода;
	\item $x=0, \Delta f(0)=0, x_n=\pi n, n\not=0, n \in Z$, --- точки разрыва 2-го рода.
\end{enumerate}
\item \begin{enumerate*}
	\item $a=0$;
	\item $a=1/3$;
	\item не существует;
	\item $a=-1$.	
\end{enumerate*}
\item \begin{enumerate*}
	\item $a=2, b=-1$;
	\item $a=1, b=-1$;
	\item не существуют;
	\item $a=1, b=\pi/2$.	
\end{enumerate*}
\setcounter{enumi}{24}
\item  \begin{enumerate}
	\item $f\circ g$ непрерывна, $g\circ f$ разрывна в точке $x=0$;
	\item $f\circ g$ разрывна в точках $x=0, x=\pm1$, $g\circ f$ непрерывна;
	\item $f\circ g$ разрывна в точке $x=-1$, $g\circ f$ разрывна в точке $x=1$;	
	\item $f\circ g$ и $g\circ f$ непрерывны.
\end{enumerate}
\setcounter{enumi}{50}
\item \begin{enumerate*}
	\item $y=\cos^2x, x\in(-\pi/2; \pi/2)$;
	\item $x=\sqrt{1+y^2}, y\in R$;
\end{enumerate*}
 \end{enumerate}

\newpage
\begin{gnuplot}
set xrange [*:*]
set yrange [*:*]
set zrange [*:*]
splot sqrt(y**2+2*x**2+2) notitle
set terminal pdfcairo
\end{gnuplot}
\\
\centering Это трехмерная поверхность $z=\sqrt{y^2+2x^2+2}$.


\end{document}
